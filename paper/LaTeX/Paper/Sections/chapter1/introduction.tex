\section{Introduction}

Brazil is a difficult country to study monetary policy shocks for two main reasons. First, it possess a history of monetary instabilities, with new currencies frequently replacing each other (see Table \hl{Table with past Brazilian currencies}). Therefore, any study that goes past 1994, the year when the Brazilian Real was created, will face a challenge to isolate the effects of the actions of the central bank from those arising from several political crises, episodes of hyperinflation, and divergent macroeconomic policies. Second, even after 1994, some statistical series had their methodology altered (e.g. industrial production has three different versions), and others are very new (e.g. the economic activity index starts in 2003\footnote{The economic activity index is calculated by the Central Bank of Brazil under the name IBC-br.}).
The inflation targeting regime was established in 1999.

VAR monthly with industrial production
    - extra variables for robustness check
    - alternative measures of activity
        - ibc-br
        - gdp growth
        - output gap (quarterly data)



\begin{figure}
    \centering
    \begin{subfigure}[b]{0.45\textwidth}
        \includegraphics[width=\textwidth]{{var1A.irf.nominal_interest_rate.inflation_adj}.eps}
        \caption{A gull}
        \label{fig:gull}
    \end{subfigure}
    ~ %add desired spacing between images, e. g. ~, \quad, \qquad, \hfill etc. 
      %(or a blank line to force the subfigure onto a new line)
    \begin{subfigure}[b]{0.45\textwidth}
        \includegraphics[width=\textwidth]{{var1B.irf.nominal_interest_rate.inflation_adj}.eps}
        \caption{A tiger}
        \label{fig:tiger}
    \end{subfigure}
    \caption{Pictures of animals}\label{fig:animals}
\end{figure}
