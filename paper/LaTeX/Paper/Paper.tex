% Preamble
\documentclass[12pt,hidelinks]{article} 
\usepackage{amsmath} % Advanced math typesetting
\usepackage[utf8]{inputenc} % Unicode support (Umlauts etc.)
\usepackage{hyperref} % Add a link to your document
\usepackage{graphicx} % Add pictures to your document
\graphicspath{ {Images/} }
%\usepackage{subfig}
\usepackage{subcaption}
\usepackage{float}
\usepackage{listings} % Source code formatting and highlighting
%\usepackage[backend=biber, citestyle=authoryear, maxbibnames=9, maxcitenames=2, mincitenames=1]{biblatex}
\usepackage[
    backend=biber,
    style=apa,
    %style=authoryear-comp,
    natbib=true,
    url=false, 
    doi=true,
    eprint=false
]{biblatex}
\usepackage[american]{babel}
\DeclareLanguageMapping{american}{american-apa}
\usepackage[a4paper, margin=1in]{geometry}
%\usepackage{indentfirst}
\usepackage{setspace}
\usepackage{csquotes}
\usepackage{datetime}
\usepackage{accents}
\usepackage{xcolor}
\usepackage{relsize}
\usepackage{pdfpages}
\usepackage{tikz}
\usepackage{titling}
\usetikzlibrary{shapes,arrows,positioning}
\usepackage{caption}
\captionsetup[figure]{labelfont=bf,font=footnotesize}
\usepackage{longtable}
\usepackage{lipsum}
\usepackage{ragged2e}
\onehalfspacing
\renewcommand{\baselinestretch}{1.5}
\setlength{\parskip}{1em}
\bibliography{library}
\numberwithin{equation}{subsection}
\AtEveryBibitem{%
  \clearfield{issn} % Remove issn
  \clearfield{doi} % Remove doi

  \ifentrytype{online}{}{% Remove url except for @online
    \clearfield{url}
  }
}
\newcommand{\myparagraph}[1]{\paragraph{#1}\mbox{}\\}
\newcommand{\hl}[1]{\colorbox{yellow}{#1}}
\setcounter{page}{3}
\setcounter{secnumdepth}{3}
\setcounter{tocdepth}{3}
\makeatletter


\title{The effects of monetary policy shocks in a small open economy}
\author{João \textsc{Dimas}} 
\newcommand{\@institution}{Université Paris 1 Panthéon-Sorbonne}
\newcommand{\@supervisor}{Professor Jean-Bernard \textsc{Chatelain}}
\newcommand{\@department}{UFR 02 Economics}
\newcommand{\@program}{M2R Financial Economics}

\begin{document}
%    \begin{titlepage}
    % \centering
    % \Huge{\@institution}

    % \begin{large}
    %     \@department

    %     \@programme

    % \end{large}
    % \vspace{80px}
    % \huge{\@title}
    % \vspace{20px}
    
    % \begin{large}
    %     \@author

    %     Supervised by \@supervisor
    %     \vfill

    %     % Bottom of the page
    %     {\large \the\year\par}

    % \end{large}

\newcommand{\HRule}{\rule{\linewidth}{0.5mm}} % Defines a new command for the horizontal lines, change thickness here

\center % Center everything on the page
 
%----------------------------------------------------------------------------------------
%   HEADING SECTIONS
%----------------------------------------------------------------------------------------

\textsc{\LARGE \@institution}\\[1.5cm] % Name of your university/college
\textsc{\Large \@department}\\[0.5cm] % Major heading such as course name
\textsc{\large \@program}\\[0.5cm] % Minor heading such as course title

%----------------------------------------------------------------------------------------
%   TITLE SECTION
%----------------------------------------------------------------------------------------

\HRule \\[0.4cm]
{ \huge \bfseries \@title}\\[0.4cm] % Title of your document
\HRule \\[1cm]
 
\large
\linespread{1.5}
Author: \\\@author
\vspace{0.3cm}
\center
\large
Supervisor: \\\@supervisor\\[2cm]

\normalsize \textbf{Abstract}
\justify \small \singlespacing
The paper will analyze the characteristics of the main existing macroprudential framework, the Basel accord, in its different versions. The different ways to measure credit, mainly, but also market risk and how it evolved to address different problems, procyclicality being the leading one.  By the same token, the complexity of designing such measures is studied; this is done referring mainly to the empirical findings about the credit risk-taking channel of monetary policy and the rational expectations critique. A further development of the work on macroprudential tightening effects during low interest rate times is suggested.
\vspace{2cm}
\center
{\large 2018}\\[3cm] % Date, change the \today to a set date if you want to be precise

\vfill % Fill the rest of the page with whitespace


\end{titlepage}
%    \tableofcontents
%    \chapter{An empirical assessment of monetary policy shocks in Brazil}
    \lipsum[1]
    \section{Data sources}
    \lipsum[1]
    \section{Variables}
    \subsection{Inflation rate}
    \input{Sections/chapter1/variables.inflation_rate.seasonality}
    \subsection{Nominal interest rate}
    \lipsum[1]
    \subsection{Output gap}
Series from IPEA described in \citet{Souza-Junior2005}
    \subsection{Real exchange rate}
Series calculated by the Central Bank of Brazil under the name \textit{Real effective exchange rate index (IPCA)}. Spanning from 1994 to 2017, this series represents the variation of Brazilian Real (rise = depreciation) against an aggregate measure of the foreign currencies of 15 trade partners weighted by their participation in Brazilian exports---adjusted by inflation.

\input{Sections/chapter1/variables.real_ex_rate.seasonality}

    \section{Baseline VAR}
    Starting from the seminal work of \citet{Sims1992}, we build a structural vector autoregressive model using Cholesky decomposition to restrict contemporaneous interactions among the variables. The recursive ordering is necessary to obtain dynamic responses from an exogenous shock---specifically, to assess the effects of a monetary policy shock on the other variables of the system.
    \subsection{Specification}
    Our first specification is composed by \(\pi_t\), \(Y_t\), \(R_t\) in quarterly frequency.
    \subsection{Diagnosis}
The baseline VAR is stable (Table \hl{table with stability test}) and does not present serially correlated errors (Table \hl{table with the serial correlation test}).
    \subsection{Robustness checks}
        \myparagraph{Using real GDP growth instead of output gap}
        \myparagraph{Adding commodities prices}
        \myparagraph{Adding real effective exchange rate}
    \subsection{Forecast performance}
        \lipsum[1]
    \subsection{Alternative scenarios}
        \lipsum[1]
    \subsection{Impulse responses}
        \lipsum[1]
    \section{Factor augmented VAR}
        \lipsum[1]
        \subsection{Specification}
            \lipsum[1]
        \subsection{Robustness checks}
            \lipsum[1]
        \subsection{Forecast performance}
            \lipsum[1]
        \subsection{Alternative scenarios}
            \lipsum[1]
        \subsection{Impulse responses}
            \lipsum[1]
%    \chapter{An agent-based model with evolving households}
    \lipsum[1]
    \section{Baseline model}
        \lipsum[1]
        \subsection{Scenarios}
            \lipsum[1]
        \subsection{Additional assumptions}
            \lipsum[1]
        \subsection{Results}
            \lipsum[1]
            \subsubsection{VAR specification}
                \lipsum[1]
            \subsubsection{Combining results of several replications}
                \lipsum[1]
            \subsubsection{Scenario 1}
                \lipsum[1]
                \myparagraph{Impulse responses}
                    \lipsum[1]
            \subsubsection{Scenario 4}
                \lipsum[1]
                \myparagraph{Impulse responses}
                    \lipsum[1]
    \section{Extension: open economy}
        \lipsum[1]
        \subsection{Scenarios}
            \lipsum[1]
        \subsection{Results}
            \lipsum[1]
            \subsection{VAR specification}
                \lipsum[1]
            \subsection{Scenario 1}
                \lipsum[1]
                \myparagraph{Impulse responses}   
                    \lipsum[1]
            \subsection{Scenario 2}
                \lipsum[1]
                \myparagraph{Impulse responses}
                    \lipsum[1]
    
%    \printbibliography
\section{Draft - Preliminary results with Brazilian data}

All VARs are stable, present heteroskedasticity, and have errors NOT normally distributed.
Some specifications do not have serial correlation, although the majority has. In those cases, adding more lags or variables did not solve the problem.

The challenge is caused by the low number of lags (4 in monthly data, 2 in quarterly data; chosen according to AIC) due to the series availability starting in 2002 for Industrial Production (monthly), 2003 for Central Bank's Economic Activity Index - IBC-br (monthly), 1995 for Output Gap (quarterly) and 1995 for real GDP growth (quarterly).

All estimations used Cholesky decomposition with economic activity (output gap, industrial production, real GDP growth, or IBC-br) ordered before Inflation rate; the nominal interest rate was ordered both firstly (assuming a lagged reaction of monetary policy) and lastly (assuming a contemporaneous reaction of monetary policy).

Below I wrote a short description of the variables. Afterwards I show all impulse responses of Inflation to a shock in Monetary policy using different specifications. I start with few variables and increment the model as a robustness check.


\myparagraph{Inflation}
I adjusted Inflation (monthly rate annualized) for seasonality using X13-ARIMA.

\myparagraph{money supply M1}
Seasonally adjusted using X13-ARIMA.

\myparagraph{Commodities prices}
No seasonality detected.

\myparagraph{Real exchange rate}
The effective rate weighted by trade share.

\myparagraph{Industrial Production}
This series is calculated by the Instituto Nacional de Geografia e Estatística (IBGE) with the name PIM-PF. I have downloaded it from the IBGE's website in monthly \% change, seasonally adjusted. Importantly, though, this series has been modified two times since its conception in 1985. The first modification, in 2004, incorporated data retrospectively up to 1991, with the older series, from 1991 to 2004, being deprecated. The second modification---current version---came to light in 2014, and incorporates data retrospectively up to 2002. As a result, I have three distinct series for different time periods : version 1 (from 1985 to 2004), version 2 (from 1991 to 2014), and version 3 (from 2002 onwards). Our estimations used the last version.

\myparagraph{Economic activity index (IBC-br)}
The IBC-br is an index calculated by the Central Bank of Brazil. It is an aggregate of 21 variables of economic activity. It is intended to be used as an anticipation of the real GDP, since the latter is measured only quarterly. I downloaded the series from the Central Bank's website in monthly level, already seasonally adjusted.

\myparagraph{Output Gap}
Quarterly data calculated by FGV. Uses an estimation of NAIRU to calculate potential GDP.

\myparagraph{Real GDP growth}
Quarterly data from IMF database. Strong seasonality detected; adjusted with X13-ARIMA.

\subsection{Industrial Production}
\begin{figure}[H]
    \centering
    \begin{subfigure}[b]{0.45\textwidth}
        \includegraphics[width=\textwidth]{/ind_prod/{var1A.irf.nominal_interest_rate.inflation_adj}.eps}
        \caption{Interest Rate ordered first; No serial correlation.}
    \end{subfigure}
    ~ 
    \begin{subfigure}[b]{0.45\textwidth}
        \includegraphics[width=\textwidth]{/ind_prod/{var1B.irf.nominal_interest_rate.inflation_adj}.eps}
        \caption{Interest Rate ordered last; No serial correlation.}
    \end{subfigure}
    \caption{VAR 1: Nominal Interest Rate, Industrial Production, Inflation. }
\end{figure}

\begin{figure}[H]
    \centering
    \begin{subfigure}[b]{0.45\textwidth}
        \includegraphics[width=\textwidth]{/ind_prod/{var2A.irf.nominal_interest_rate.inflation_adj}.eps}
        \caption{Interest rate ordered first. Serial correlation detected.}
    \end{subfigure}
    ~ 
    \begin{subfigure}[b]{0.45\textwidth}
        \includegraphics[width=\textwidth]{/ind_prod/{var2B.irf.nominal_interest_rate.inflation_adj}.eps}
        \caption{Interest rate ordered last. Serial correlation detected.}
    \end{subfigure}
    \caption{VAR 2: Nominal Interest Rate, M1 (log diff.), Industrial Production, Inflation}
\end{figure}

\begin{figure}[H]
    \centering
    \begin{subfigure}[b]{0.45\textwidth}
        \includegraphics[width=\textwidth]{/ind_prod/{var3A.irf.nominal_interest_rate.inflation_adj}.eps}
        \caption{Interest rate ordered first. Serial correlation detected.}
    \end{subfigure}
    ~ 
    \begin{subfigure}[b]{0.45\textwidth}
        \includegraphics[width=\textwidth]{/ind_prod/{var3B.irf.nominal_interest_rate.inflation_adj}.eps}
        \caption{Interest rate ordered last. Serial correlation detected.}
    \end{subfigure}
    \caption{VAR 3: Nominal Interest Rate, Real Exchange Rate, M1 (log diff.), Industrial Production, Inflation}
\end{figure}

\begin{figure}[H]
    \centering
    \begin{subfigure}[b]{0.45\textwidth}
        \includegraphics[width=\textwidth]{/ind_prod/{var4A.irf.nominal_interest_rate.inflation_adj}.eps}
        \caption{Interest rate ordered first. Serial correlation detected.}
    \end{subfigure}
    ~ 
    \begin{subfigure}[b]{0.45\textwidth}
        \includegraphics[width=\textwidth]{/ind_prod/{var4B.irf.nominal_interest_rate.inflation_adj}.eps}
        \caption{Interest rate ordered last. Serial correlation detected.}
    \end{subfigure}
    \caption{VAR 4: Nominal Interest Rate, Real Exchange Rate, Commodity Index (log diff.), M1 (log diff.), Industrial Production, Inflation}
\end{figure}
\section{Economic activity measured by the Central Bank's Economic Activity Index (IBC-br)}
\begin{figure}[H]
    \centering
    \begin{subfigure}[b]{0.45\textwidth}
        \includegraphics[width=\textwidth]{/ibcbr/{var1A.irf.nominal_interest_rate.inflation_adj}.eps}
        \caption{Interest Rate ordered first; No serial correlation.}
    \end{subfigure}
    ~ 
    \begin{subfigure}[b]{0.45\textwidth}
        \includegraphics[width=\textwidth]{/ibcbr/{var1B.irf.nominal_interest_rate.inflation_adj}.eps}
        \caption{Interest Rate ordered last; No serial correlation.}
    \end{subfigure}
    \caption{VAR 1: Nominal Interest Rate, IBC-br, Inflation. }
\end{figure}


\begin{figure}[H]
    \centering
    \begin{subfigure}[b]{0.45\textwidth}
        \includegraphics[width=\textwidth]{/ibcbr/{var2A.irf.nominal_interest_rate.inflation_adj}.eps}
        \caption{Interest rate ordered first. Serial correlation detected.}
    \end{subfigure}
    ~ 
    \begin{subfigure}[b]{0.45\textwidth}
        \includegraphics[width=\textwidth]{/ibcbr/{var2B.irf.nominal_interest_rate.inflation_adj}.eps}
        \caption{Interest rate ordered last. Serial correlation detected.}
    \end{subfigure}
    \caption{VAR 2: Nominal Interest Rate, M1 (log diff.), IBC-br, Inflation}
\end{figure}

\begin{figure}[H]
    \centering
    \begin{subfigure}[b]{0.45\textwidth}
        \includegraphics[width=\textwidth]{/ibcbr/{var3A.irf.nominal_interest_rate.inflation_adj}.eps}
        \caption{Interest rate ordered first. Serial correlation detected.}
    \end{subfigure}
    ~ 
    \begin{subfigure}[b]{0.45\textwidth}
        \includegraphics[width=\textwidth]{/ibcbr/{var3B.irf.nominal_interest_rate.inflation_adj}.eps}
        \caption{Interest rate ordered last. Serial correlation detected.}
    \end{subfigure}
    \caption{VAR 3: Nominal Interest Rate, Real Exchange Rate, M1 (log diff.), IBC-br, Inflation}
\end{figure}

\begin{figure}[H]
    \centering
    \begin{subfigure}[b]{0.45\textwidth}
        \includegraphics[width=\textwidth]{/ibcbr/{var4A.irf.nominal_interest_rate.inflation_adj}.eps}
        \caption{Interest rate ordered first. Serial correlation detected.}
    \end{subfigure}
    ~ 
    \begin{subfigure}[b]{0.45\textwidth}
        \includegraphics[width=\textwidth]{/ibcbr/{var4B.irf.nominal_interest_rate.inflation_adj}.eps}
        \caption{Interest rate ordered last. Serial correlation detected.}
    \end{subfigure}
    \caption{VAR 4: Nominal Interest Rate, Real Exchange Rate, Commodity Index (log diff.), M1 (log diff.), IBC-br, Inflation}
\end{figure}
\subsection{Output Gap}

\begin{figure}[H]
    \centering
    \begin{subfigure}[b]{0.45\textwidth}
        \includegraphics[width=\textwidth]{/ibcbr/{var1A.irf.nominal_interest_rate.inflation_adj}.eps}
        \caption{Interest Rate ordered first. NO Serial correlation.}
    \end{subfigure} 
    ~ 
    \begin{subfigure}[b]{0.45\textwidth}
        \includegraphics[width=\textwidth]{/ibcbr/{var1B.irf.nominal_interest_rate.inflation_adj}.eps}
        \caption{Interest Rate ordered last. NO Serial correlation.}
    \end{subfigure}
    \caption{VAR 1: Nominal Interest Rate, Output Gap, Inflation. }
\end{figure}
\subsection{Real GDP growth rate}

\begin{figure}[H]
    \centering
    \begin{subfigure}[b]{0.45\textwidth}
        \includegraphics[width=\textwidth]{/ibcbr/{var2A.irf.nominal_interest_rate.inflation_adj}.eps}
        \caption{Interest rate ordered first. Serial correlation detected.}
    \end{subfigure}
    ~ 
    \begin{subfigure}[b]{0.45\textwidth}
        \includegraphics[width=\textwidth]{/ibcbr/{var2B.irf.nominal_interest_rate.inflation_adj}.eps}
        \caption{Interest rate ordered last. Serial correlation detected.}
    \end{subfigure}
    \caption{VAR 1: Nominal Interest Rate, Real GDP growth rate, Inflation}
\end{figure}
 \end{document}